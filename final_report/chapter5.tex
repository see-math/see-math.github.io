\chapter{Discussion On The Achievements}

This project has been a topic that was completely new for all of us team members. With interests on learning web development, knowing how they operate and with a motive to construct a tool, that not only supports us as a learners base, but has the ability to be used for practical learning and making concepts plain and understandable. The following work has been achieved as a result of this project:
\begin{itemize}
	\item Gained a deep understanding of web frameworks, communication with the server and user-interaction.
	\item Built proficiency in scripting for websites as well as developed a hint for web design.
	\item A completely working graphing function, that plots the user defined curves has been integrated.
	\item Root finding through bisection method and Newton-Raphson method has been successfully implemented.
	\item Implemented the functionality to message the developers for reviews and complaints.
\end{itemize}


\section{{\bf{Features}}}

See-math promises to be a great visualization tool for various numerical method algorithms as well as the construction of finite automata machines. Till date, the following features have been integrated on our website:
\begin{itemize}
	\item A message form for interaction.
	\item Grapher : Plots the given function.
	\item Bisection Method
	\item Newton-Raphson Method
\end{itemize}

\section{{\bf{Limitations}}}

See-math is a website meant for visualizing mathematcal problems and gaining better understanding of the problem. The goal of the website is to solve and visualize as much mathematical problems as possible but current only some of the problems reated to numerical methods such as bisection methon and newton raphson method can be fully calculated and visualized.

\section{{\bf{Future Enhancements}}}

The current see-math lacks in comaprisoin to the ideal visualization website meant to solve numerous mathematical problems. Various methods are yet to be added and many enhancements are yet to be made and in the near future various features and new method will slowly but surely be included.
